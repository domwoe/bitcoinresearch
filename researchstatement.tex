\documentclass[a4paper, 12pt]{scrartcl}
\usepackage[utf8]{inputenc}
\usepackage{amsmath,amsfonts}
\usepackage{scrpage2}
\usepackage{scrtime}
\usepackage{setspace}
\setlength{\parindent}{0em}
\setlength{\parskip}{1em}

\title{Research Statement}
\author{Dominic W\"orner \\ Department of Management, Technology and Economics \\ ETH Zurich}

\begin{document}

\maketitle

I've a background in physics and I'm pursuing a PhD in information management at the Swiss Federal Institute of Technology in Zurich. Moreover, I'm part of the the Bosch Internet of Things and Services Lab, a joint initiative of the Robert Bosch GmbH and the University of St. Gallen. My overarching research theme is capturing value in the Internet of Things. My research has always been driven by building prototypes of information systems that can be evaluated in real-world settings. I've started in the smart home domain. Together with a colleague I built several prototypes of heating and room climate information and control systems following the design science research paradigm. The main question I've been perusing is how can an information system support residents in not wasting heating energy.
Surprisingly, the greatest values our field study participants experienced, were remote control and the ability to easily set schedules for individual rooms, instead of more technically advanced features like geolocation-based heating controls. Preliminary work has been published in technical and design science research conferences, and I'm currently finalizing a concluding paper that will be submitted to the journal Energy \& Buildings.

In addition, I investigated the value of room climate sensor data for inferring individual room occupancy information using Hidden Markov Models. This is just one example which highlights the importance for interoperability and the sharing of data between Internet of Things applications.

This is also one of the reasons I got excited about Bitcoin, and cryptocurrencies in general. Bitcoin is open and inclusive. It can be seen as a protocol for transferring packets of value, analogous to TCP/IP for transferring packets of data. Bitcoin could not only be the economic layer for the Internet of Things, but also help to keep it decentralized and secure. I co-authored a paper that discussed the characteristics of Bitcoin regarding the Sensing as a Service pattern. In its simplest form, this is the ad-hoc provision of sensing capabilities or real-time sensor data in exchange for money. I also demonstrated the concept with a prototype at UbiComp 2015. Such a pattern however is in need of a nanopayment scheme on which I'm working with colleagues from distributed computing.

My vision is that connected devices will be economic agents representing either the interest of companies, end users, or even their own. Thereby, the major asset of connected devices is their ability to digitize their physical and social environment and their ability to offer these data on a transparent, world-wide market. This would incentivze sharing of data and provide the foundation to capture the value of the Internet of Things across multiple stakeholders. However, data has the property to be reproducible and might contain personal information. Thus, a simple sharing or selling of data is not always appropriate. That is why I'm so interested in visiting the MIT Media Lab, and working on the Enigma project in particular. 


\end{document}