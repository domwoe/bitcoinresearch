\documentclass[a4paper, 12pt]{scrartcl}

\usepackage[utf8]{inputenc}
\usepackage{amsmath,amsfonts}
\usepackage{scrpage2}
\usepackage{scrtime}
\usepackage{hyperref}
\usepackage[xindy]{glossaries}

\makeglossaries




\begin{document}
\newglossaryentry{smartcontract}
{
  name={Smart contract},
  description={A smart contract is a computerized transaction protocol that executes the terms of a contract. The general objectives of a smart contract design are to satisfy common contractual conditions (such as payment terms, liens, confidentiality, and even enforcement),minimize exceptions both malicious and accidental, and minimize the need for trusted intermediaries. Related economic goals include lowering fraud loss, arbitration and enforcement costs, and other transaction costs. (Nick Szabo, 1994)\\

  Smart contracts are contracts as program code, where the terms of the contract are enforced by the logic of the program's execution. In a series of steps, from the basic metaphor of contracts as board games, through the nature of contract-created derivative rights, to compositions of games to turn assets into capital, we explain how smart contracts can resolve the conflict - gaining the benefits of global transferability without sacrificing local knowledge. (Mark S. Miller, 2003) \\

  Smart contracts are computer protocols that facilitate, verify, execute and enforce the terms of a commercial agreement. (Tim Swanson, 2014) \\

  A smart contract is an event-driven program, with state, which runs on a replicated, shared ledger and which can take custody over assets on that ledger (Richard Gendal Brown, 2015)\\

  A smart contract is a simple rules engine; cryptographically assured business logic that has the ability to execute and move value. (Tim Swanson, 2015)\\

  A smart contract is cryptographically verifiable execution of code over cryptographically verifiable data (Casey Kuhlman, 2015)\\

  A smart contract is a computer program that directly controls digital assets and which is run in such an environment that it can be trusted to faithfully execute (Vitalik Buterin, 2015)}
}

\newglossaryentry{agent} {
	name={Autonomous agent},
	description={An autonomous agent is a system situated within and a part of an environment that senses that environment and acts on it, over time, in pursuit of its own agenda and so as to affect what it senses in the future. }
}
\printglossaries
\end{document}

@incollection{
year={1997},
isbn={978-3-540-62507-0},
booktitle={Intelligent Agents III Agent Theories, Architectures, and Languages},
volume={1193},
series={Lecture Notes in Computer Science},
editor={M\"uller, J\"org P. and Wooldridge, Michael J. and Jennings, Nicholas R.},
doi={10.1007/BFb0013570},
title={Is It an agent, or just a program?: A taxonomy for autonomous agents},
url={http://dx.doi.org/10.1007/BFb0013570},
publisher={Springer Berlin Heidelberg},
author={Franklin, Stan and Graesser, Art},
pages={21-35},
language={English}
}
